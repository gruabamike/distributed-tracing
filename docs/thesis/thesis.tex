% tasks
% [ ] general strucutre
% [ ] first capter


\documentclass[a4paper,12p]{article}

% Language setting
\usepackage[ngerman]{babel}


% Set page size and margins
% Replace `letterpaper' with `a4paper' for UK/EU standard size
\usepackage[a4paper,top=2cm,bottom=2cm,left=3cm,right=3cm,marginparwidth=1.75cm]{geometry}

% Useful packages

\usepackage{graphicx}
\usepackage[colorlinks=true, allcolors=blue]{hyperref}

\title{Verteiltes Tracing in einer Microservice-Architektur}
\author{Michael Gruber}
\date{}

\begin{document}
\maketitle


% TODO: Titelblatt

% TODO: Eidesstattliche Erklärung

% Erklärung
% Ich erkläre eidesstattlich, dass ich die vorliegende Arbeit selbstständig und ohne fremde Hilfe verfasst, andere als die angegebenen Quellen nicht benutzt und die den benutzten Quellen entnommenen Stellen als solche gekennzeichnet habe. Die Arbeit wurde bisher in gleicher oder ähnlicher Form keiner anderen Prüfungsbehörde vorgelegt.

% Die vorliegende, gedruckte Masterarbeit ist identisch zu dem elektronisch übermittelten Textdokument.

% Datum                                 Unterschrift

% TODO: Danksagung


\begin{abstract}
Kurzfassung (DE)

% - Trend geht Richtung serverless (evtl. aktuelle Statistik?)-> verteiltese Tracing umso wichtiger
% - Viele verschiedene Tracing-Tools bzw. Anbieter; bisher ohne Standard, nun aber OpenTelemetry
% - Bestandteile von Open Source Instrumentation (OpenTelemetry)
% - Praktischer Einsatz von OpenTelemetry im .NET Microservice-Umfeld unter Verwendung von asynchroner Kommunikation

\end{abstract}

\begin{abstract}
    Abstract (EN)
\end{abstract}

% TODO: Inhaltsverzeichnis

\section{Einführung}
\label{sec:einführung}

\subsection{Motivation}
\label{sec:motivation}

Begrifflichkeiten wie "Cloud-Computing" und "SaaS" sind heutzutage in aller Munde. Immer mehr Softwareanbieter setzen bei Ihren Produkten vermehrt auf Cloud-Lösungen, anstatt den altbekannten "OnPremise"-Lösungen.

\subsection{Zielsetzung}

\section{Grundlagen}
In diesem Kapitel werden die Grundlagen und Konzepte, auf denen diese Arbeit basiert näher erläutert, um so ein besseres Verständnis über die nachfolgenden Kapiteln zu erhalten.

\subsection{Microservice-Architektur}


\subsection{Monitoring vs. Beobachtbarkeit}

Beobachtbarkeit wird häufig als Synonym für Monitoring verwendet~\cite{Hadfield2022}. Während der Begriff der Beobachtbarkeit in der Informatik vor allem aufgrund des Trends zu verteilten Systemen immer mehr an Bedeutung gewann, hat sich im Gegensatz dazu Monitoring schon länger etabliert. Genauer betrachtet existieren jedoch wichtige Unterschiede in der Terminologie.

Monitoring beschreibt eine aktive Tätigkeit, bei der wir das System überwachen~\cite[p.~310]{Newman2021}. Der interne Zustand wird auf Basis von Protokollen, Messungen oder Beobachtungen (meist periodisch) kontrolliert. Monitoring ist reaktiv; sobald ein Problem eintritt, erfolgt eine schnelle Reaktion auf den Zwischenfall. Beispiel dafür wäre ein Dienst, welcher nicht mehr erreichbar ist oder vom erwarteten Zustand abweicht. In fortgeschritteneren Stufen des Monitoring beschäftigt man sich bereits im Vorfeld mit potentiellen Fehlerquellen sowie einem dazugehörigen Alarmierungssystem, um schnellstmöglich aufmerksam gemacht zu werden.

Die Beobachtbarkeit ist dahingegen mehr als eine Eigenschaft des Systems zu betrachten, indem die internen Zustände aus den externen Ausgaben abgeleitet werden können~\cite[p.~309]{Newman2021}. Die Herausforderung besteht meist darin, diese externen Ausgaben zu erstellen und mit den richtigen Werkzeugen Erkenntnisse daraus zu ziehen. Beobachtbarkeit ist proaktiv; aus dem transparenten Verhalten resultiert ein besseres Verständnis des Gesamtsystems. Infolgedessen können zu jedem Zeitpunkt gezielt Fragen an das System gerichtet werden. Gerade bei verteilten Systemen entpuppt sich dies als klarer Vorteil, da meist Problemstellungen auftreten, die man im Vorhinein nicht definieren kann. Je besser die Beobachtbarkeit eines Systems ist, desto schneller können die Problemstellungen nachvollzogen und die Ursache auf einen bestimmten Bereich eingegrenzt werden.

\subsection{Logs}

System erfasst Ereignisse in Form von Text, strukturierte Daten oder binäre 

werden ausgegeben oder für zukünftige Analysen gespeichert.

Im Kontext von Beobachtbarkeit meist zentrales Logging, wo Daten von einzelnen Services an ein zentrales System übermittelt werden, um eine gesamtheitliche Suche und Analyse durchführen zu können

Arten von Logs
    - Text
    - Strutkurierte Daten in From von JSON
    - binäre Daten (protbuf)

- Bestandteile von Logs
    - Zeitstempel
    - standard
    
- Log Levels (DEBUG, INFO, WARNING, ERROR, FATAL)

correlation ID
fix fertige Stacks (ELK)

primäre challenge ist, die großen Logdaten in einem kosteneffektiven Weg zu persistieren.
Die meisten Loginformationen sind nicht von bedeutung, aber jene für das Auffinden eines Fehlers sind umso kostbarer.

\subsection{Metriken}

Metriken sind Sammlungen von Statistiken über vereinzelte Dienste, die es Entwickler und Betreiber ermöglichen, deren grundsätzliches Verhalten undieser Dienste zu verstehen

Beispiele
* Request Rate, Average Duration, Queue Size, Number of Requests, Number of Errors, number of active Users

Zeitseriendatenbank für änderungen über einen Zeitraum
Veränderungen können mit anderen zufälligen Ereignissen korreliert werden, welches bei der Fehlerbehebung hilfreich sein kann.

In Produktionsumgebungen werden Metriken üblicherweise je nach Kontext periodisch aggregiert
Metriken müssen schnell aggregiert und visualisiert werden - nahe an Echtzeit, um schnell genug reagieren zu können

Metriken werden meist Labels zugewiesen (Version, Host, Datencenter etc.)
dies kann Betreiber helfen, ein Problem auf eine bestimmte Version oder Host einzugrenzen

Zwei Kategorien von Metriken
* Counter
    - numerischer Zähler, welcher die Anzahl an aufgetretenen Ereignissen zählt und meist erhöht wird
    - einfaches aggregieren von Änderungen an einem Zähler aus mehreren Quellen
    - Änderungsrate berechnen, indem die Werte eines Zählers zu verschiedenen Zeitpunkten betrachtet werden (Anfrage pro min/sec)

* Gauges
    - Metrik, die den Zustand eines Subsystemes als numerischen Wert zu einem bestimmten Zeitpunkt beschreibt
    - zB Durchschnittszeit für einen Request, Größe einer Queue, aktuelle Anzahl an Benutzer
    - Wert wird nicht inkrementiert, sondern gesetzt/aktualisiert
    - schwieriger für Aggregation, aber dafür leicht für MIN, MAX, AVG
    - in Kombination mit Histogramm gut geeignet
    

\subsection{Traces}

\subsection{Die drei Säulen der Beobachtbarkeit}
Besonders im Umfeld von Cloud-native Anwendungen formen die Begriffe Logs, Metriken und Traces die drei Säulen der Beobachtbarkeit.

NewRelic hat sogar das Acronym MELT geprägt.

\subsection{.NET}

\subsection{Messaging}

\subsection{Docker}

\subsection{Docker-Compose}

% TODO: Service Mesh, Sidecar vs Agent, Collectors, Gateway??


\section{Bestandteile eines verteilten Tracingsystems}

\subsection{Instrumentierung}

\subsection{Deployment}

\subsection{Leistungsüberwachung}

\section{OpenTelemetry: Die Zukunft der Instrumentierung}

\section{Hands-On: Verteiltes Tracing in .NET mit OpenTelemetry}

\section{Ausblick}

% TODO: ggf. Abbildungsverzeichnis

% TODO: ggf. Abkürzungsverzeichnis

% TODO: ggf. Glossar
%   - Cloud-native Anwendung

% TODO: Bibliographie (Unterteilung Literatur und Online-Quellen)


\bibliographystyle{alpha}
\bibliography{references}

\end{document}